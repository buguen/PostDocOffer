%%%%%%%%%%%%%%%%%%%%%%%%%%%%%%%%%%%%%%%%%
% Long Lined Cover Letter
% LaTeX Template
% Version 1.0 (1/6/13)
%
% This template has been downloaded from:
% http://www.LaTeXTemplates.com
%
% Original author:
% Matthew J. Miller
% http://www.matthewjmiller.net/howtos/customized-cover-letter-scripts/
%
% License:
% CC BY-NC-SA 3.0 (http://creativecommons.org/licenses/by-nc-sa/3.0/)
%
%%%%%%%%%%%%%%%%%%%%%%%%%%%%%%%%%%%%%%%%%

%----------------------------------------------------------------------------------------
%	PACKAGES AND OTHER DOCUMENT CONFIGURATIONS
%----------------------------------------------------------------------------------------

\documentclass[10pt,stdletter,dateno,sigleft]{newlfm} % Extra options: 'sigleft' for a left-aligned signature, 'stdletternofrom' to remove the from address, 'letterpaper' for US letter paper - consult the newlfm class manual for more options

\usepackage{charter} % Use the Charter font for the document text
\usepackage{url}
\usepackage{hyperref}
\newsavebox{\Luiuc}\sbox{\Luiuc}{\parbox[b]{1.75in}{\vspace{0.5in}
\includegraphics[width=1.2\linewidth]{logo.png}}} % Company/institution logo at the top left of the page
\makeletterhead{Uiuc}{\Lheader{\usebox{\Luiuc}}}

\newlfmP{sigsize=50pt} % Slightly decrease the height of the signature field
%\newlfmP{addrfromphone} % Print a phone number under the sender's address
%\newlfmP{addrfromemail} % Print an email address under the sender's address
%\PhrPhone{Phone} % Customize the "Telephone" text
\PhrEmail{Post Doctoral Offer at IETR} % Customize the "E-mail" text

\lthUiuc % Print the company/institution logo

%----------------------------------------------------------------------------------------
%	YOUR NAME AND CONTACT INFORMATION
%----------------------------------------------------------------------------------------

%\namefrom{Post Doctoral Offer at IETR} % Name

%\addrfrom{
%\today\\[12pt] % Date
%123 Broadway \\ % Address
%City, State 12345
%}

%\phonefrom{(000) 111-1111} % Phone number

%
%----------------------------------------------------------------------------------------
%	ADDRESSEE AND GREETING/CLOSING
%----------------------------------------------------------------------------------------

%\greetto{Dear Mrs. Smith,} % Greeting text
%\closeline{Sincerely yours,} % Closing text

%\nameto{Mrs. Jane Smith} % Addressee of the letter above the to address

%\addrto{
%Recruitment Officer \\ % To address
%The Corporation \\
%123 Pleasant Lane \\
%City, State 12345
%}

%----------------------------------------------------------------------------------------

\begin{document}
\begin{newlfm}
\vspace{-4cm}
\begin{center}
{ \Large Post Doctoral Offer on 5G Radio Channel Modeling } 
\end{center}

{\bf General Context}

This offer holds in the framework of the CominLabs project {\tt \href{http://www.m5hstia.cominlabs.ueb.eu/}{M5HESTIA}} on the topic of
mmW Multi­user Massive MIMO Hybrid Equipments for Sounding, Transmissions and HW ImplementAtion.

mmW is an extremely attractive enabler for the next generation(5G) mobile communications.

Operation in these high frequency bands offers very large bandwidths
is one of the simplest ways to increase system capacity, but also
leads to device miniaturisation thanks to the small related wavelength.
In this context, M­MIMO systems, with up to hundreds of radiating
elements at the radio Access Point (AP), are intended to achieve
very high data rates for multiple users sharing the same spectrum at the
same time, with low power consumption  thanks to the use of specific analogue/digital precoding
techniques.

The M5HESTIA project has two main objectives:

\begin{itemize}
	\item The first one consists in providing a precise and realistic model for the M­MIMO channel in the 60­GHz band. Indeed, most of already­ available theoretical results are based
	on very simple or theoretical models that do not take into
	account channel propagation and electronics specificities of very
	high frequencies such as, for example, electromagnetic
	coupling, power issues and other impairments. 

	\item The second main objective is the
		hardware(HW)implementation of a Massive-MIMO system
		deployed at the base station (or AP) in a multi­user
		context based on SDMA (spatial division multiple
		access) technique. Indeed, the larger the number of
		transmit antennas on infrastructure side,the larger the
	number of users who can be simultaneously served in the same time-frequency slot.
\end{itemize}
	For this purpose,analogue and/or digital beamforming techniques
	(implemented to improve link budgets at 60 GHz,one of the
	most important critical points at these frequencies) and
	digital precoding techniques (SVD decomposition,ZF, MMSE, time
	reversal, conjugatebeamforming, etc.) will be studied and combined
	(leading to a hybrid analogue and digital system).

In particular, M5HESTIA project will optimise these  different
analogue and digital processing techniques by using the
proposed channel model and by taking  into account real antennas
characteristics that will be also developed in the project.


{\bf Description of the research project:} 
The current offer is for working on
the channel modeling task of the M5HESTIA project. 
The main task will be to propose and implement a mmW M­MIMO channel model based on extensive channel
measurements campaign realized during the project. The approach will be based on deterministic
and/or quasi deterministic approaches aiming to produce test vector for PHY layer simulation. The
effect of the environment on the channel angular structure and its coherence over time will be
investigated. The impact of the diffused energy will also be investigated from the measurement data
and depolarization estimation.


The model should be able  to take as much as possible the effectof the base­band/RF architecture.
The proposed channel model (or simulator) should allow addressing a wide range antenna
specifications and architectures in order to figure out which is the best overall architecture for mmW.
In particular this includes the antenna coupling effect and specification of the separation of baseband
and RF functions number of baseband fluxes number of antennas,and switching or phasing network.
The effect of the environment on the channel angular structure (in particular elevation spread) and its
coherence over time will be investigated. ​
The model should be scalable toward larger problems 
(typically more antennas) while being validated over a smaller set of antennas from the measurement
campaign performed during the project.
The deterministic or quasi­ deterministic simulator will implement, jointly with channel simulations,
the vehicular and human mobility in order to assess the effect on the overall system performance.
The effects of flashing rays caused by the  blockage from subject and object  mobility will be
investigated.

{\bf Expected candidate Profile }: 

Ph.D in propagation modeling or digital communications with a good
proficiency in mathematics and statistics and computational implementation of channel models.
Python language proficiency would be very appreciated.

\begin{itemize}
\item {\bf Location}:
IETR­ UMR 6164­ Campus de Beaulieu Rennes­ France
\item {\bf Expected start date}:
April 2017 
\item {\bf duration}: 18 months
\item {\bf Salary}:
     2000 Euros net/month
\item {\bf Application}:
Send a CV, a motivation letter and a recommendation letter
to Pr Bernard UGUEN­
bernard.uguen@univ­rennes1.fr
,@BernardUguen
\end{itemize}


\end{newlfm}
\end{document}
